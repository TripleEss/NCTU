\begin{abstractzh}
\par 在這資訊爆炸的時代,每天有大量的數位本文被產出。為了從這些資料中獲取有用的訊息,文字探勘成了當前的熱門議題,而本文分類便是其中的重要任務之一。
\par 在本論文中,我們提出一個用於多類別本文分類的向量空間模型,詞向量CopeOpi vectors。我們將用於中文情感分析的CopeOpi scores,擴充至能夠用於多類別本文分類且無語言限制的CopeOpi vectors。
\par 我們測試CopeOpi vectors於英文及中文的情感分析及主題分類問題,並與幾個常用於本文分類的特徵向量進行比較,將這些特徵向量套用至不同的機器學習演算法。實驗結果顯示CopeOpi vectors能夠用更小的向量長度與更短的訓練時間,達到與其他特徵向量同樣水平的分類成果。 CopeOpi vectors是適用於多類別本文分類,兼具效果與效率的詞向量。
~\newline
~\newline
\textbf{關鍵字:}本文分類;向量空間模型;詞向量
\end{abstractzh}